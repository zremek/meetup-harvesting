\documentclass[]{article}
\usepackage{lmodern}
\usepackage{amssymb,amsmath}
\usepackage{ifxetex,ifluatex}
\usepackage{fixltx2e} % provides \textsubscript
\ifnum 0\ifxetex 1\fi\ifluatex 1\fi=0 % if pdftex
  \usepackage[T1]{fontenc}
  \usepackage[utf8]{inputenc}
\else % if luatex or xelatex
  \ifxetex
    \usepackage{mathspec}
  \else
    \usepackage{fontspec}
  \fi
  \defaultfontfeatures{Ligatures=TeX,Scale=MatchLowercase}
\fi
% use upquote if available, for straight quotes in verbatim environments
\IfFileExists{upquote.sty}{\usepackage{upquote}}{}
% use microtype if available
\IfFileExists{microtype.sty}{%
\usepackage{microtype}
\UseMicrotypeSet[protrusion]{basicmath} % disable protrusion for tt fonts
}{}
\usepackage[margin=1in]{geometry}
\usepackage{hyperref}
\hypersetup{unicode=true,
            pdftitle={Meetupy data science w Polsce. Stan na 2019-06-04},
            pdfborder={0 0 0},
            breaklinks=true}
\urlstyle{same}  % don't use monospace font for urls
\usepackage{color}
\usepackage{fancyvrb}
\newcommand{\VerbBar}{|}
\newcommand{\VERB}{\Verb[commandchars=\\\{\}]}
\DefineVerbatimEnvironment{Highlighting}{Verbatim}{commandchars=\\\{\}}
% Add ',fontsize=\small' for more characters per line
\usepackage{framed}
\definecolor{shadecolor}{RGB}{248,248,248}
\newenvironment{Shaded}{\begin{snugshade}}{\end{snugshade}}
\newcommand{\AlertTok}[1]{\textcolor[rgb]{0.94,0.16,0.16}{#1}}
\newcommand{\AnnotationTok}[1]{\textcolor[rgb]{0.56,0.35,0.01}{\textbf{\textit{#1}}}}
\newcommand{\AttributeTok}[1]{\textcolor[rgb]{0.77,0.63,0.00}{#1}}
\newcommand{\BaseNTok}[1]{\textcolor[rgb]{0.00,0.00,0.81}{#1}}
\newcommand{\BuiltInTok}[1]{#1}
\newcommand{\CharTok}[1]{\textcolor[rgb]{0.31,0.60,0.02}{#1}}
\newcommand{\CommentTok}[1]{\textcolor[rgb]{0.56,0.35,0.01}{\textit{#1}}}
\newcommand{\CommentVarTok}[1]{\textcolor[rgb]{0.56,0.35,0.01}{\textbf{\textit{#1}}}}
\newcommand{\ConstantTok}[1]{\textcolor[rgb]{0.00,0.00,0.00}{#1}}
\newcommand{\ControlFlowTok}[1]{\textcolor[rgb]{0.13,0.29,0.53}{\textbf{#1}}}
\newcommand{\DataTypeTok}[1]{\textcolor[rgb]{0.13,0.29,0.53}{#1}}
\newcommand{\DecValTok}[1]{\textcolor[rgb]{0.00,0.00,0.81}{#1}}
\newcommand{\DocumentationTok}[1]{\textcolor[rgb]{0.56,0.35,0.01}{\textbf{\textit{#1}}}}
\newcommand{\ErrorTok}[1]{\textcolor[rgb]{0.64,0.00,0.00}{\textbf{#1}}}
\newcommand{\ExtensionTok}[1]{#1}
\newcommand{\FloatTok}[1]{\textcolor[rgb]{0.00,0.00,0.81}{#1}}
\newcommand{\FunctionTok}[1]{\textcolor[rgb]{0.00,0.00,0.00}{#1}}
\newcommand{\ImportTok}[1]{#1}
\newcommand{\InformationTok}[1]{\textcolor[rgb]{0.56,0.35,0.01}{\textbf{\textit{#1}}}}
\newcommand{\KeywordTok}[1]{\textcolor[rgb]{0.13,0.29,0.53}{\textbf{#1}}}
\newcommand{\NormalTok}[1]{#1}
\newcommand{\OperatorTok}[1]{\textcolor[rgb]{0.81,0.36,0.00}{\textbf{#1}}}
\newcommand{\OtherTok}[1]{\textcolor[rgb]{0.56,0.35,0.01}{#1}}
\newcommand{\PreprocessorTok}[1]{\textcolor[rgb]{0.56,0.35,0.01}{\textit{#1}}}
\newcommand{\RegionMarkerTok}[1]{#1}
\newcommand{\SpecialCharTok}[1]{\textcolor[rgb]{0.00,0.00,0.00}{#1}}
\newcommand{\SpecialStringTok}[1]{\textcolor[rgb]{0.31,0.60,0.02}{#1}}
\newcommand{\StringTok}[1]{\textcolor[rgb]{0.31,0.60,0.02}{#1}}
\newcommand{\VariableTok}[1]{\textcolor[rgb]{0.00,0.00,0.00}{#1}}
\newcommand{\VerbatimStringTok}[1]{\textcolor[rgb]{0.31,0.60,0.02}{#1}}
\newcommand{\WarningTok}[1]{\textcolor[rgb]{0.56,0.35,0.01}{\textbf{\textit{#1}}}}
\usepackage{graphicx,grffile}
\makeatletter
\def\maxwidth{\ifdim\Gin@nat@width>\linewidth\linewidth\else\Gin@nat@width\fi}
\def\maxheight{\ifdim\Gin@nat@height>\textheight\textheight\else\Gin@nat@height\fi}
\makeatother
% Scale images if necessary, so that they will not overflow the page
% margins by default, and it is still possible to overwrite the defaults
% using explicit options in \includegraphics[width, height, ...]{}
\setkeys{Gin}{width=\maxwidth,height=\maxheight,keepaspectratio}
\IfFileExists{parskip.sty}{%
\usepackage{parskip}
}{% else
\setlength{\parindent}{0pt}
\setlength{\parskip}{6pt plus 2pt minus 1pt}
}
\setlength{\emergencystretch}{3em}  % prevent overfull lines
\providecommand{\tightlist}{%
  \setlength{\itemsep}{0pt}\setlength{\parskip}{0pt}}
\setcounter{secnumdepth}{0}
% Redefines (sub)paragraphs to behave more like sections
\ifx\paragraph\undefined\else
\let\oldparagraph\paragraph
\renewcommand{\paragraph}[1]{\oldparagraph{#1}\mbox{}}
\fi
\ifx\subparagraph\undefined\else
\let\oldsubparagraph\subparagraph
\renewcommand{\subparagraph}[1]{\oldsubparagraph{#1}\mbox{}}
\fi

%%% Use protect on footnotes to avoid problems with footnotes in titles
\let\rmarkdownfootnote\footnote%
\def\footnote{\protect\rmarkdownfootnote}

%%% Change title format to be more compact
\usepackage{titling}

% Create subtitle command for use in maketitle
\providecommand{\subtitle}[1]{
  \posttitle{
    \begin{center}\large#1\end{center}
    }
}

\setlength{\droptitle}{-2em}

  \title{Meetupy data science w Polsce. Stan na 2019-06-04}
    \pretitle{\vspace{\droptitle}\centering\huge}
  \posttitle{\par}
    \author{}
    \preauthor{}\postauthor{}
    \date{}
    \predate{}\postdate{}
  

\begin{document}
\maketitle

To jest analiza 86 polskich grup meetupowych, które miały wpisany temat
``data science'' na dzień 2019-06-04.

Chcę pokazać meetupy w \textbf{przestrzeni} (1) i \textbf{czasie} (2).
Idzie to do rozdziału, gdzie pokazuję też inne rzeczy poza meetupami,
stąd wykres 1.1.

\begin{center}\rule{0.5\linewidth}{\linethickness}\end{center}

Dane pobrałem przez API meetup.com za pomocą pakietu w Pythonie. Kod i
pliki .json są w /meetup\_api/.

Pliki źródłowe zapisane do R w 1\_read\_json.R

Złączenie danych zrobione w 2\_join.R

\hypertarget{przestrzen}{%
\subsection{1 - Przestrzeń}\label{przestrzen}}

Kod w 3\_space\_viz.R

Wykresy są zrobione pod strukturę i formatowanie pracy doktorskiej
(rozdziały, A4, Times New Roman).

\hypertarget{lokalizacja-dziaalnosci-swiata-spoecznego-data-science-w-polsce---firmy-studia-meetupy}{%
\subsubsection{1.1. Lokalizacja działalności świata społecznego data
science w Polsce - firmy, studia,
meetupy}\label{lokalizacja-dziaalnosci-swiata-spoecznego-data-science-w-polsce---firmy-studia-meetupy}}

\begin{Shaded}
\begin{Highlighting}[]
\KeywordTok{plot}\NormalTok{(groups_firms_edu)}
\end{Highlighting}
\end{Shaded}

\includegraphics{all_viz_files/figure-latex/unnamed-chunk-1-1.pdf}

\hypertarget{suma-czonkow-grup-meetupowych-data-science-i-ilosc-czonkow-na-grupe-w-podziale-na-12-polskich-miast}{%
\subsubsection{1.2. Suma członków grup meetupowych data science i ilość
członków na grupę w podziale na 12 polskich
miast}\label{suma-czonkow-grup-meetupowych-data-science-i-ilosc-czonkow-na-grupe-w-podziale-na-12-polskich-miast}}

\begin{Shaded}
\begin{Highlighting}[]
\KeywordTok{plot}\NormalTok{(map_bar)}
\end{Highlighting}
\end{Shaded}

\includegraphics{all_viz_files/figure-latex/unnamed-chunk-2-1.pdf}

\hypertarget{ilosc-czonkow-86-grup-meetupowych-w-podziale-na-dwanascie-polskich-miast}{%
\subsubsection{1.3. Ilość członków 86 grup meetupowych w podziale na
dwanaście polskich
miast}\label{ilosc-czonkow-86-grup-meetupowych-w-podziale-na-dwanascie-polskich-miast}}

\begin{Shaded}
\begin{Highlighting}[]
\KeywordTok{plot}\NormalTok{(lolli)}
\end{Highlighting}
\end{Shaded}

\begin{verbatim}
## Warning: Transformation introduced infinite values in continuous y-axis
\end{verbatim}

\includegraphics{all_viz_files/figure-latex/unnamed-chunk-3-1.pdf}

\hypertarget{czas}{%
\subsection{2 - Czas}\label{czas}}

\hypertarget{powstawanie-grup-meetupowych-data-science-w-latach-2012---2019-do-czerwca}{%
\subsubsection{2.1. Powstawanie grup meetupowych data science w latach
2012 - 2019 (do
czerwca)}\label{powstawanie-grup-meetupowych-data-science-w-latach-2012---2019-do-czerwca}}

\begin{Shaded}
\begin{Highlighting}[]
\KeywordTok{plot}\NormalTok{(n_cumsum_year)}
\end{Highlighting}
\end{Shaded}

\begin{verbatim}
## Warning: Removed 1 rows containing missing values (geom_point).
\end{verbatim}

\begin{verbatim}
## Warning: Removed 1 rows containing missing values (geom_segment).
\end{verbatim}

\includegraphics{all_viz_files/figure-latex/unnamed-chunk-4-1.pdf}

\hypertarget{termin-powstania-86-grup-meetupowych-data-science-oraz-termin-zaplanowanego-spotkania-w-podziale-na-dwanascie-polskich-miast}{%
\subsubsection{2.2. Termin powstania 86 grup meetupowych data science
oraz termin zaplanowanego spotkania w podziale na dwanaście polskich
miast}\label{termin-powstania-86-grup-meetupowych-data-science-oraz-termin-zaplanowanego-spotkania-w-podziale-na-dwanascie-polskich-miast}}

\begin{Shaded}
\begin{Highlighting}[]
\KeywordTok{plot}\NormalTok{(next_event)}
\end{Highlighting}
\end{Shaded}

\begin{verbatim}
## Warning: Removed 69 rows containing missing values (geom_segment).
\end{verbatim}

\begin{verbatim}
## Warning: Removed 69 rows containing missing values (geom_point).
\end{verbatim}

\includegraphics{all_viz_files/figure-latex/unnamed-chunk-5-1.pdf}

\hypertarget{powstawanie-86-grup-meetupowych-data-science-w-podziale-na-dwanascie-polskich-miast-i-w-zaleznosc-od-czasu}{%
\subsubsection{2.3. Powstawanie 86 grup meetupowych data science w
podziale na dwanaście polskich miast i w zależność od
czasu}\label{powstawanie-86-grup-meetupowych-data-science-w-podziale-na-dwanascie-polskich-miast-i-w-zaleznosc-od-czasu}}

\begin{Shaded}
\begin{Highlighting}[]
\KeywordTok{plot}\NormalTok{(box_dif)}
\end{Highlighting}
\end{Shaded}

\includegraphics{all_viz_files/figure-latex/unnamed-chunk-6-1.pdf}

\hypertarget{wybrane-terminy-w-nazwach-86-grup-meetupowych-data-science-wedug-roku-powstania-grupy}{%
\subsubsection{2.4. Wybrane terminy w nazwach 86 grup meetupowych data
science według roku powstania
grupy}\label{wybrane-terminy-w-nazwach-86-grup-meetupowych-data-science-wedug-roku-powstania-grupy}}

\begin{Shaded}
\begin{Highlighting}[]
\KeywordTok{plot}\NormalTok{(names_time)}
\end{Highlighting}
\end{Shaded}

\includegraphics{all_viz_files/figure-latex/unnamed-chunk-7-1.pdf}


\end{document}
